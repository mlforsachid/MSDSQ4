\documentclass[]{article}
\usepackage{lmodern}
\usepackage{amssymb,amsmath}
\usepackage{ifxetex,ifluatex}
\usepackage{fixltx2e} % provides \textsubscript
\ifnum 0\ifxetex 1\fi\ifluatex 1\fi=0 % if pdftex
  \usepackage[T1]{fontenc}
  \usepackage[utf8]{inputenc}
\else % if luatex or xelatex
  \ifxetex
    \usepackage{mathspec}
  \else
    \usepackage{fontspec}
  \fi
  \defaultfontfeatures{Ligatures=TeX,Scale=MatchLowercase}
\fi
% use upquote if available, for straight quotes in verbatim environments
\IfFileExists{upquote.sty}{\usepackage{upquote}}{}
% use microtype if available
\IfFileExists{microtype.sty}{%
\usepackage{microtype}
\UseMicrotypeSet[protrusion]{basicmath} % disable protrusion for tt fonts
}{}
\usepackage[margin=1in]{geometry}
\usepackage{hyperref}
\hypersetup{unicode=true,
            pdftitle={Data-621 Homework-2},
            pdfborder={0 0 0},
            breaklinks=true}
\urlstyle{same}  % don't use monospace font for urls
\usepackage{color}
\usepackage{fancyvrb}
\newcommand{\VerbBar}{|}
\newcommand{\VERB}{\Verb[commandchars=\\\{\}]}
\DefineVerbatimEnvironment{Highlighting}{Verbatim}{commandchars=\\\{\}}
% Add ',fontsize=\small' for more characters per line
\usepackage{framed}
\definecolor{shadecolor}{RGB}{248,248,248}
\newenvironment{Shaded}{\begin{snugshade}}{\end{snugshade}}
\newcommand{\AlertTok}[1]{\textcolor[rgb]{0.94,0.16,0.16}{#1}}
\newcommand{\AnnotationTok}[1]{\textcolor[rgb]{0.56,0.35,0.01}{\textbf{\textit{#1}}}}
\newcommand{\AttributeTok}[1]{\textcolor[rgb]{0.77,0.63,0.00}{#1}}
\newcommand{\BaseNTok}[1]{\textcolor[rgb]{0.00,0.00,0.81}{#1}}
\newcommand{\BuiltInTok}[1]{#1}
\newcommand{\CharTok}[1]{\textcolor[rgb]{0.31,0.60,0.02}{#1}}
\newcommand{\CommentTok}[1]{\textcolor[rgb]{0.56,0.35,0.01}{\textit{#1}}}
\newcommand{\CommentVarTok}[1]{\textcolor[rgb]{0.56,0.35,0.01}{\textbf{\textit{#1}}}}
\newcommand{\ConstantTok}[1]{\textcolor[rgb]{0.00,0.00,0.00}{#1}}
\newcommand{\ControlFlowTok}[1]{\textcolor[rgb]{0.13,0.29,0.53}{\textbf{#1}}}
\newcommand{\DataTypeTok}[1]{\textcolor[rgb]{0.13,0.29,0.53}{#1}}
\newcommand{\DecValTok}[1]{\textcolor[rgb]{0.00,0.00,0.81}{#1}}
\newcommand{\DocumentationTok}[1]{\textcolor[rgb]{0.56,0.35,0.01}{\textbf{\textit{#1}}}}
\newcommand{\ErrorTok}[1]{\textcolor[rgb]{0.64,0.00,0.00}{\textbf{#1}}}
\newcommand{\ExtensionTok}[1]{#1}
\newcommand{\FloatTok}[1]{\textcolor[rgb]{0.00,0.00,0.81}{#1}}
\newcommand{\FunctionTok}[1]{\textcolor[rgb]{0.00,0.00,0.00}{#1}}
\newcommand{\ImportTok}[1]{#1}
\newcommand{\InformationTok}[1]{\textcolor[rgb]{0.56,0.35,0.01}{\textbf{\textit{#1}}}}
\newcommand{\KeywordTok}[1]{\textcolor[rgb]{0.13,0.29,0.53}{\textbf{#1}}}
\newcommand{\NormalTok}[1]{#1}
\newcommand{\OperatorTok}[1]{\textcolor[rgb]{0.81,0.36,0.00}{\textbf{#1}}}
\newcommand{\OtherTok}[1]{\textcolor[rgb]{0.56,0.35,0.01}{#1}}
\newcommand{\PreprocessorTok}[1]{\textcolor[rgb]{0.56,0.35,0.01}{\textit{#1}}}
\newcommand{\RegionMarkerTok}[1]{#1}
\newcommand{\SpecialCharTok}[1]{\textcolor[rgb]{0.00,0.00,0.00}{#1}}
\newcommand{\SpecialStringTok}[1]{\textcolor[rgb]{0.31,0.60,0.02}{#1}}
\newcommand{\StringTok}[1]{\textcolor[rgb]{0.31,0.60,0.02}{#1}}
\newcommand{\VariableTok}[1]{\textcolor[rgb]{0.00,0.00,0.00}{#1}}
\newcommand{\VerbatimStringTok}[1]{\textcolor[rgb]{0.31,0.60,0.02}{#1}}
\newcommand{\WarningTok}[1]{\textcolor[rgb]{0.56,0.35,0.01}{\textbf{\textit{#1}}}}
\usepackage{graphicx,grffile}
\makeatletter
\def\maxwidth{\ifdim\Gin@nat@width>\linewidth\linewidth\else\Gin@nat@width\fi}
\def\maxheight{\ifdim\Gin@nat@height>\textheight\textheight\else\Gin@nat@height\fi}
\makeatother
% Scale images if necessary, so that they will not overflow the page
% margins by default, and it is still possible to overwrite the defaults
% using explicit options in \includegraphics[width, height, ...]{}
\setkeys{Gin}{width=\maxwidth,height=\maxheight,keepaspectratio}
\IfFileExists{parskip.sty}{%
\usepackage{parskip}
}{% else
\setlength{\parindent}{0pt}
\setlength{\parskip}{6pt plus 2pt minus 1pt}
}
\setlength{\emergencystretch}{3em}  % prevent overfull lines
\providecommand{\tightlist}{%
  \setlength{\itemsep}{0pt}\setlength{\parskip}{0pt}}
\setcounter{secnumdepth}{0}
% Redefines (sub)paragraphs to behave more like sections
\ifx\paragraph\undefined\else
\let\oldparagraph\paragraph
\renewcommand{\paragraph}[1]{\oldparagraph{#1}\mbox{}}
\fi
\ifx\subparagraph\undefined\else
\let\oldsubparagraph\subparagraph
\renewcommand{\subparagraph}[1]{\oldsubparagraph{#1}\mbox{}}
\fi

%%% Use protect on footnotes to avoid problems with footnotes in titles
\let\rmarkdownfootnote\footnote%
\def\footnote{\protect\rmarkdownfootnote}

%%% Change title format to be more compact
\usepackage{titling}

% Create subtitle command for use in maketitle
\providecommand{\subtitle}[1]{
  \posttitle{
    \begin{center}\large#1\end{center}
    }
}

\setlength{\droptitle}{-2em}

  \title{Data-621 Homework-2}
    \pretitle{\vspace{\droptitle}\centering\huge}
  \posttitle{\par}
    \author{}
    \preauthor{}\postauthor{}
    \date{}
    \predate{}\postdate{}
  

\begin{document}
\maketitle

\hypertarget{download-the-classification-output-data-set-attached-in-blackboard-to-the-assignment.}{%
\paragraph{1. Download the classification output data set (attached in
Blackboard to the
assignment).}\label{download-the-classification-output-data-set-attached-in-blackboard-to-the-assignment.}}

\begin{itemize}
\tightlist
\item
  \textbf{Load Data}
\end{itemize}

\begin{Shaded}
\begin{Highlighting}[]
\NormalTok{input.df =}\StringTok{ }\KeywordTok{read.csv}\NormalTok{(}\StringTok{"D:/MSDS/MSDSQ4/Data621/HW-2/classification-output-data.csv"}\NormalTok{, }\DataTypeTok{stringsAsFactors =} \OtherTok{FALSE}\NormalTok{)}
\end{Highlighting}
\end{Shaded}

\begin{itemize}
\tightlist
\item
  \textbf{View Data}
\end{itemize}

\begin{Shaded}
\begin{Highlighting}[]
\KeywordTok{head}\NormalTok{(input.df)}
\end{Highlighting}
\end{Shaded}

\begin{verbatim}
##   pregnant glucose diastolic skinfold insulin  bmi pedigree age class
## 1        7     124        70       33     215 25.5    0.161  37     0
## 2        2     122        76       27     200 35.9    0.483  26     0
## 3        3     107        62       13      48 22.9    0.678  23     1
## 4        1      91        64       24       0 29.2    0.192  21     0
## 5        4      83        86       19       0 29.3    0.317  34     0
## 6        1     100        74       12      46 19.5    0.149  28     0
##   scored.class scored.probability
## 1            0         0.32845226
## 2            0         0.27319044
## 3            0         0.10966039
## 4            0         0.05599835
## 5            0         0.10049072
## 6            0         0.05515460
\end{verbatim}

\hypertarget{the-data-set-has-three-key-columns-we-will-use}{%
\paragraph{2. The data set has three key columns we will
use:}\label{the-data-set-has-three-key-columns-we-will-use}}

\begin{itemize}
\tightlist
\item
  \textbf{class:} the actual class for the observation
\item
  \textbf{scored.class:} the predicted class for the observation (based
  on a threshold of 0.5)
\item
  \textbf{scored.probability:} the predicted probability of success for
  the observation
\end{itemize}

\hypertarget{use-the-table-function-to-get-the-raw-confusion-matrix-for-this-scored-dataset.-make-sure-you-understand-the-output.-in-particular-do-the-rows-represent-the-actual-or-predicted-class-the-columns}{%
\paragraph{Use the table() function to get the raw confusion matrix for
this scored dataset. Make sure you understand the output. In particular,
do the rows represent the actual or predicted class? The
columns?}\label{use-the-table-function-to-get-the-raw-confusion-matrix-for-this-scored-dataset.-make-sure-you-understand-the-output.-in-particular-do-the-rows-represent-the-actual-or-predicted-class-the-columns}}

\begin{Shaded}
\begin{Highlighting}[]
\NormalTok{conf.mat =}\StringTok{ }\KeywordTok{table}\NormalTok{(input.df}\OperatorTok{$}\NormalTok{class, input.df}\OperatorTok{$}\NormalTok{scored.class)}
\NormalTok{conf.mat}
\end{Highlighting}
\end{Shaded}

\begin{verbatim}
##    
##       0   1
##   0 119   5
##   1  30  27
\end{verbatim}

\begin{itemize}
\item
  \textbf{Above table shows the confusion matrix. In the above table
  rows represents the Actual class and columns represent the predicted
  class}
\item
  \textbf{Create a dataframe for above metrics}
\end{itemize}

\begin{Shaded}
\begin{Highlighting}[]
\NormalTok{create.metrics =}\StringTok{ }\ControlFlowTok{function}\NormalTok{(actclass, predclass)}
\NormalTok{\{}
\NormalTok{  TN =}\StringTok{ }\KeywordTok{sum}\NormalTok{(actclass }\OperatorTok{==}\StringTok{ }\DecValTok{0} \OperatorTok{&}\StringTok{ }\NormalTok{predclass }\OperatorTok{==}\StringTok{ }\DecValTok{0}\NormalTok{)}
\NormalTok{  FP =}\StringTok{ }\KeywordTok{sum}\NormalTok{(actclass }\OperatorTok{==}\StringTok{ }\DecValTok{0} \OperatorTok{&}\StringTok{ }\NormalTok{predclass }\OperatorTok{==}\StringTok{ }\DecValTok{1}\NormalTok{)}
\NormalTok{  FN =}\StringTok{ }\KeywordTok{sum}\NormalTok{(actclass }\OperatorTok{==}\StringTok{ }\DecValTok{1} \OperatorTok{&}\StringTok{ }\NormalTok{predclass }\OperatorTok{==}\StringTok{ }\DecValTok{0}\NormalTok{)}
\NormalTok{  TP =}\StringTok{ }\KeywordTok{sum}\NormalTok{(actclass }\OperatorTok{==}\StringTok{ }\DecValTok{1} \OperatorTok{&}\StringTok{ }\NormalTok{predclass }\OperatorTok{==}\StringTok{ }\DecValTok{1}\NormalTok{)}
\NormalTok{  metrics.df =}\StringTok{ }\KeywordTok{data.frame}\NormalTok{(}\DataTypeTok{TN=}\NormalTok{TN, }\DataTypeTok{FN =}\NormalTok{ FN, }\DataTypeTok{TP =}\NormalTok{ TP, }\DataTypeTok{FP =}\NormalTok{ FP)}
  \KeywordTok{return}\NormalTok{(metrics.df)}
  
\NormalTok{\}}

\NormalTok{metrics.df =}\StringTok{ }\KeywordTok{create.metrics}\NormalTok{(input.df[,}\StringTok{'class'}\NormalTok{], input.df[,}\StringTok{'scored.class'}\NormalTok{])}
\end{Highlighting}
\end{Shaded}

\hypertarget{write-a-function-that-takes-the-data-set-as-a-dataframe-with-actual-and-predicted-classifications-identifiedand-returns-the-accuracy-of-the-predictions.}{%
\paragraph{3. Write a function that takes the data set as a dataframe,
with actual and predicted classifications identified,and returns the
accuracy of the
predictions.}\label{write-a-function-that-takes-the-data-set-as-a-dataframe-with-actual-and-predicted-classifications-identifiedand-returns-the-accuracy-of-the-predictions.}}

\textbf{Accuracy = (TP + TN)/ (TP + FP + TN + FN)}

\begin{itemize}
\tightlist
\item
  \textbf{Function to calculate Accuracy}
\end{itemize}

\begin{Shaded}
\begin{Highlighting}[]
\NormalTok{calc.accuracy =}\StringTok{ }\ControlFlowTok{function}\NormalTok{(df)}
\NormalTok{\{}
\NormalTok{  Accuracy =}\StringTok{ }\NormalTok{(df}\OperatorTok{$}\NormalTok{TP }\OperatorTok{+}\StringTok{ }\NormalTok{df}\OperatorTok{$}\NormalTok{TN)}\OperatorTok{/}\StringTok{ }\NormalTok{(df}\OperatorTok{$}\NormalTok{TP }\OperatorTok{+}\StringTok{ }\NormalTok{df}\OperatorTok{$}\NormalTok{FP }\OperatorTok{+}\StringTok{ }\NormalTok{df}\OperatorTok{$}\NormalTok{TN }\OperatorTok{+}\StringTok{ }\NormalTok{df}\OperatorTok{$}\NormalTok{FN)}
  \KeywordTok{return}\NormalTok{(Accuracy)}
\NormalTok{\}}

\KeywordTok{print}\NormalTok{(}\KeywordTok{paste}\NormalTok{(}\StringTok{'Accuracy = '}\NormalTok{, }\KeywordTok{calc.accuracy}\NormalTok{(metrics.df)))}
\end{Highlighting}
\end{Shaded}

\begin{verbatim}
## [1] "Accuracy =  0.806629834254144"
\end{verbatim}

\hypertarget{write-a-function-that-takes-the-data-set-as-a-dataframe-with-actual-and-predicted-classifications-identifiedand-returns-the-classification-error-rate-of-the-predictions.}{%
\paragraph{4. Write a function that takes the data set as a dataframe,
with actual and predicted classifications identified,and returns the
classification error rate of the
predictions.}\label{write-a-function-that-takes-the-data-set-as-a-dataframe-with-actual-and-predicted-classifications-identifiedand-returns-the-classification-error-rate-of-the-predictions.}}

\textbf{Classification Error Rate = (FP + FN)/(TP + FP + TN + FN)}

\begin{itemize}
\tightlist
\item
  \textbf{Function to calculate Classification Error}
\end{itemize}

\begin{Shaded}
\begin{Highlighting}[]
\NormalTok{calc.error =}\StringTok{ }\ControlFlowTok{function}\NormalTok{(df)}
\NormalTok{\{}
\NormalTok{  Error =}\StringTok{ }\NormalTok{(df}\OperatorTok{$}\NormalTok{FP }\OperatorTok{+}\StringTok{ }\NormalTok{df}\OperatorTok{$}\NormalTok{FN)}\OperatorTok{/}\StringTok{ }\NormalTok{(df}\OperatorTok{$}\NormalTok{TP }\OperatorTok{+}\StringTok{ }\NormalTok{df}\OperatorTok{$}\NormalTok{FP }\OperatorTok{+}\StringTok{ }\NormalTok{df}\OperatorTok{$}\NormalTok{TN }\OperatorTok{+}\StringTok{ }\NormalTok{df}\OperatorTok{$}\NormalTok{FN)}
  \KeywordTok{return}\NormalTok{(Error)}
\NormalTok{\}}

\KeywordTok{print}\NormalTok{(}\KeywordTok{paste}\NormalTok{(}\StringTok{'Classification Error = '}\NormalTok{, }\KeywordTok{calc.error}\NormalTok{(metrics.df)))}
\end{Highlighting}
\end{Shaded}

\begin{verbatim}
## [1] "Classification Error =  0.193370165745856"
\end{verbatim}

\hypertarget{verify-that-you-get-an-accuracy-and-an-error-rate-that-sums-to-one.}{%
\paragraph{Verify that you get an accuracy and an error rate that sums
to
one.}\label{verify-that-you-get-an-accuracy-and-an-error-rate-that-sums-to-one.}}

\begin{Shaded}
\begin{Highlighting}[]
\KeywordTok{print}\NormalTok{(}\KeywordTok{calc.accuracy}\NormalTok{(metrics.df) }\OperatorTok{+}\StringTok{ }\KeywordTok{calc.error}\NormalTok{(metrics.df))}
\end{Highlighting}
\end{Shaded}

\begin{verbatim}
## [1] 1
\end{verbatim}

\begin{itemize}
\tightlist
\item
  \textbf{Above calculation proves that Accuracy and Error rate sums to
  one}
\end{itemize}

\hypertarget{write-a-function-that-takes-the-data-set-as-a-dataframe-with-actual-and-predicted-classifications-identified-and-returns-the-precision-of-the-predictions.}{%
\paragraph{5. Write a function that takes the data set as a dataframe,
with actual and predicted classifications identified, and returns the
precision of the
predictions.}\label{write-a-function-that-takes-the-data-set-as-a-dataframe-with-actual-and-predicted-classifications-identified-and-returns-the-precision-of-the-predictions.}}

\textbf{Precision = TP/(TP + FP)}

\begin{Shaded}
\begin{Highlighting}[]
\NormalTok{calc.precision =}\StringTok{ }\ControlFlowTok{function}\NormalTok{(df)}
\NormalTok{\{}
\NormalTok{  Precision =}\StringTok{ }\NormalTok{(df}\OperatorTok{$}\NormalTok{TP)}\OperatorTok{/}\StringTok{ }\NormalTok{(df}\OperatorTok{$}\NormalTok{TP }\OperatorTok{+}\StringTok{ }\NormalTok{df}\OperatorTok{$}\NormalTok{FP)}
  \KeywordTok{return}\NormalTok{(Precision)}
\NormalTok{\}}

\KeywordTok{print}\NormalTok{(}\KeywordTok{paste}\NormalTok{(}\StringTok{'Precision = '}\NormalTok{, }\KeywordTok{calc.precision}\NormalTok{(metrics.df)))}
\end{Highlighting}
\end{Shaded}

\begin{verbatim}
## [1] "Precision =  0.84375"
\end{verbatim}

\hypertarget{write-a-function-that-takes-the-data-set-as-a-dataframe-with-actual-and-predicted-classifications-identified-and-returns-the-sensitivity-of-the-predictions.-sensitivity-is-also-known-as-recall.}{%
\paragraph{6. Write a function that takes the data set as a dataframe,
with actual and predicted classifications identified, and returns the
sensitivity of the predictions. Sensitivity is also known as
recall.}\label{write-a-function-that-takes-the-data-set-as-a-dataframe-with-actual-and-predicted-classifications-identified-and-returns-the-sensitivity-of-the-predictions.-sensitivity-is-also-known-as-recall.}}

\textbf{Sensitivity = TP/(TP + FN)}

\begin{Shaded}
\begin{Highlighting}[]
\NormalTok{calc.sensitivity =}\StringTok{ }\ControlFlowTok{function}\NormalTok{(df)}
\NormalTok{\{}
\NormalTok{  Sensitivity =}\StringTok{ }\NormalTok{(df}\OperatorTok{$}\NormalTok{TP)}\OperatorTok{/}\StringTok{ }\NormalTok{(df}\OperatorTok{$}\NormalTok{TP }\OperatorTok{+}\StringTok{ }\NormalTok{df}\OperatorTok{$}\NormalTok{FN)}
  \KeywordTok{return}\NormalTok{(Sensitivity)}
\NormalTok{\}}

\KeywordTok{print}\NormalTok{(}\KeywordTok{paste}\NormalTok{(}\StringTok{'Sensitivity = '}\NormalTok{, }\KeywordTok{calc.sensitivity}\NormalTok{(metrics.df)))}
\end{Highlighting}
\end{Shaded}

\begin{verbatim}
## [1] "Sensitivity =  0.473684210526316"
\end{verbatim}

\hypertarget{write-a-function-that-takes-the-data-set-as-a-dataframe-with-actual-and-predicted-classifications-identified-and-returns-the-specificity-of-the-predictions.}{%
\paragraph{7. Write a function that takes the data set as a dataframe,
with actual and predicted classifications identified, and returns the
specificity of the
predictions.}\label{write-a-function-that-takes-the-data-set-as-a-dataframe-with-actual-and-predicted-classifications-identified-and-returns-the-specificity-of-the-predictions.}}

\textbf{Specificity = TN/(TN + FP)}

\begin{Shaded}
\begin{Highlighting}[]
\NormalTok{calc.specificity =}\StringTok{ }\ControlFlowTok{function}\NormalTok{(df)}
\NormalTok{\{}
\NormalTok{  Specificity =}\StringTok{ }\NormalTok{(df}\OperatorTok{$}\NormalTok{TN)}\OperatorTok{/}\StringTok{ }\NormalTok{(df}\OperatorTok{$}\NormalTok{TN }\OperatorTok{+}\StringTok{ }\NormalTok{df}\OperatorTok{$}\NormalTok{FP)}
  \KeywordTok{return}\NormalTok{(Specificity)}
\NormalTok{\}}

\KeywordTok{print}\NormalTok{(}\KeywordTok{paste}\NormalTok{(}\StringTok{'Specificity = '}\NormalTok{, }\KeywordTok{calc.specificity}\NormalTok{(metrics.df)))}
\end{Highlighting}
\end{Shaded}

\begin{verbatim}
## [1] "Specificity =  0.959677419354839"
\end{verbatim}

\hypertarget{write-a-function-that-takes-the-data-set-as-a-dataframe-with-actual-and-predicted-classifications-identified-and-returns-the-f1-score-of-the-predictions.}{%
\paragraph{8. Write a function that takes the data set as a dataframe,
with actual and predicted classifications identified, and returns the F1
score of the
predictions.}\label{write-a-function-that-takes-the-data-set-as-a-dataframe-with-actual-and-predicted-classifications-identified-and-returns-the-f1-score-of-the-predictions.}}

\textbf{F1 Score = (2 * Precision * Sensitivity)/ (Precision +
Sensitivity)}

\begin{Shaded}
\begin{Highlighting}[]
\NormalTok{calc.f1score =}\StringTok{ }\ControlFlowTok{function}\NormalTok{(df)}
\NormalTok{\{}
\NormalTok{  precision =}\StringTok{ }\KeywordTok{calc.precision}\NormalTok{(df)}
\NormalTok{  sensitivity =}\StringTok{ }\KeywordTok{calc.sensitivity}\NormalTok{(df)}
\NormalTok{  f1.score =}\StringTok{ }\NormalTok{(}\DecValTok{2} \OperatorTok{*}\StringTok{ }\NormalTok{precision }\OperatorTok{*}\StringTok{ }\NormalTok{sensitivity)}\OperatorTok{/}\NormalTok{(precision }\OperatorTok{+}\StringTok{ }\NormalTok{sensitivity)}
  \KeywordTok{return}\NormalTok{(f1.score)}
\NormalTok{\}}

\KeywordTok{print}\NormalTok{(}\KeywordTok{paste}\NormalTok{(}\StringTok{'F1 Score = '}\NormalTok{, }\KeywordTok{calc.f1score}\NormalTok{(metrics.df)))}
\end{Highlighting}
\end{Shaded}

\begin{verbatim}
## [1] "F1 Score =  0.606741573033708"
\end{verbatim}

\hypertarget{before-we-move-on-lets-consider-a-question-that-was-asked-what-are-the-bounds-on-the-f1-score-showthat-the-f1-score-will-always-be-between-0-and-1.}{%
\paragraph{9. Before we move on, let's consider a question that was
asked: What are the bounds on the F1 score? Showthat the F1 score will
always be between 0 and
1.}\label{before-we-move-on-lets-consider-a-question-that-was-asked-what-are-the-bounds-on-the-f1-score-showthat-the-f1-score-will-always-be-between-0-and-1.}}

\begin{itemize}
\item
  \textbf{F1 score is calculated based on Precision and Sensitivity.
  Precision is nothing but how many are true positives out of identified
  positives and sensitivity is nothing but how many true positives are
  identified out of total available positives. Based on these
  definitions and above formulae we can conclude that both precision and
  sensitivity values can range from 0 to 1}
\item
  \textbf{Below is the simple simulation of F1 score range when
  precision and sensitivity values varies from 0 to 1}
\end{itemize}

\begin{Shaded}
\begin{Highlighting}[]
\NormalTok{precision =}\StringTok{ }\KeywordTok{c}\NormalTok{(}\FloatTok{0.001}\NormalTok{, }\FloatTok{0.1}\NormalTok{, }\FloatTok{0.5}\NormalTok{, }\FloatTok{0.9}\NormalTok{, }\DecValTok{1}\NormalTok{)}
\NormalTok{recall =}\StringTok{ }\KeywordTok{c}\NormalTok{(}\FloatTok{0.001}\NormalTok{, }\FloatTok{0.1}\NormalTok{, }\FloatTok{0.5}\NormalTok{, }\FloatTok{0.9}\NormalTok{, }\DecValTok{1}\NormalTok{)}

\NormalTok{f1.score =}\StringTok{ }\NormalTok{(}\DecValTok{2}\OperatorTok{*}\NormalTok{precision }\OperatorTok{*}\StringTok{ }\NormalTok{recall)}\OperatorTok{/}\NormalTok{(precision }\OperatorTok{+}\StringTok{ }\NormalTok{recall)}
\KeywordTok{print}\NormalTok{(f1.score)}
\end{Highlighting}
\end{Shaded}

\begin{verbatim}
## [1] 0.001 0.100 0.500 0.900 1.000
\end{verbatim}

\begin{itemize}
\tightlist
\item
  \textbf{Above simulation shows that when precision and sensitivity
  values varies from 0 to 1 F1 Score takes a range of values from 0 to
  1. From the above simulation we can conclude that F1 score value can
  be within 0 to 1 range, with 0 indicating poor model fit and 1
  indicating best model fit}
\end{itemize}

\hypertarget{write-a-function-that-generates-an-roc-curve-from-a-data-set-with-a-true-classification-column-class-in-ourexample-and-a-probability-column-scored.probability-in-our-example.-your-function-should-return-a-list-that-includes-the-plot-of-the-roc-curve-and-a-vector-that-contains-the-calculated-area-under-the-curveauc.-note-that-i-recommend-using-a-sequence-of-thresholds-ranging-from-0-to-1-at-0.01-intervals.}{%
\paragraph{10. Write a function that generates an ROC curve from a data
set with a true classification column (class in ourexample) and a
probability column (scored.probability in our example). Your function
should return a list that includes the plot of the ROC curve and a
vector that contains the calculated area under the curve(AUC). Note that
I recommend using a sequence of thresholds ranging from 0 to 1 at 0.01
intervals.}\label{write-a-function-that-generates-an-roc-curve-from-a-data-set-with-a-true-classification-column-class-in-ourexample-and-a-probability-column-scored.probability-in-our-example.-your-function-should-return-a-list-that-includes-the-plot-of-the-roc-curve-and-a-vector-that-contains-the-calculated-area-under-the-curveauc.-note-that-i-recommend-using-a-sequence-of-thresholds-ranging-from-0-to-1-at-0.01-intervals.}}

\begin{Shaded}
\begin{Highlighting}[]
\NormalTok{simple_auc <-}\StringTok{ }\ControlFlowTok{function}\NormalTok{(TPR, FPR)\{}
  \CommentTok{# inputs already sorted, best scores first }
\NormalTok{  dFPR <-}\StringTok{ }\KeywordTok{c}\NormalTok{(}\KeywordTok{diff}\NormalTok{(FPR), }\DecValTok{0}\NormalTok{)}
\NormalTok{  dTPR <-}\StringTok{ }\KeywordTok{c}\NormalTok{(}\KeywordTok{diff}\NormalTok{(TPR), }\DecValTok{0}\NormalTok{)}
\NormalTok{  auc =}\StringTok{ }\KeywordTok{sum}\NormalTok{(TPR }\OperatorTok{*}\StringTok{ }\NormalTok{dFPR) }\OperatorTok{+}\StringTok{ }\KeywordTok{sum}\NormalTok{(dTPR }\OperatorTok{*}\StringTok{ }\NormalTok{dFPR)}\OperatorTok{/}\DecValTok{2}
  \KeywordTok{return}\NormalTok{(auc)}
\NormalTok{\}}
\NormalTok{calc.roc =}\StringTok{ }\ControlFlowTok{function}\NormalTok{()}
\NormalTok{\{}
  \KeywordTok{library}\NormalTok{(ggplot2)}
\NormalTok{  threshold =}\StringTok{ }\KeywordTok{seq}\NormalTok{(}\DecValTok{0}\NormalTok{,}\DecValTok{1}\NormalTok{,}\FloatTok{0.01}\NormalTok{)}
\NormalTok{  class =}\StringTok{ }\NormalTok{input.df}\OperatorTok{$}\NormalTok{class}
\NormalTok{  spec =}\StringTok{ }\KeywordTok{c}\NormalTok{()}
\NormalTok{  sens =}\StringTok{ }\KeywordTok{c}\NormalTok{()}
  \ControlFlowTok{for}\NormalTok{(t }\ControlFlowTok{in}\NormalTok{ threshold)}
\NormalTok{  \{}
\NormalTok{    scored.class =}\StringTok{ }\KeywordTok{ifelse}\NormalTok{(input.df}\OperatorTok{$}\NormalTok{scored.probability }\OperatorTok{>}\StringTok{ }\NormalTok{t, }\DecValTok{1}\NormalTok{, }\DecValTok{0}\NormalTok{)}
\NormalTok{    df =}\StringTok{ }\KeywordTok{data.frame}\NormalTok{(}\DataTypeTok{class =}\NormalTok{ class, }\DataTypeTok{scored.class =}\NormalTok{ scored.class)}
    
\NormalTok{    metrics.df =}\StringTok{ }\KeywordTok{create.metrics}\NormalTok{(df[,}\StringTok{'class'}\NormalTok{], df[,}\StringTok{'scored.class'}\NormalTok{])}
\NormalTok{    spec =}\StringTok{ }\KeywordTok{c}\NormalTok{(spec, }\KeywordTok{calc.specificity}\NormalTok{(metrics.df))}
\NormalTok{    sens =}\StringTok{ }\KeywordTok{c}\NormalTok{(sens, }\KeywordTok{calc.sensitivity}\NormalTok{(metrics.df))}
\NormalTok{  \}}
  
\NormalTok{  plt =}\StringTok{ }\NormalTok{ggplot2}\OperatorTok{::}\KeywordTok{qplot}\NormalTok{(}\DecValTok{1}\OperatorTok{-}\NormalTok{spec, sens, }\DataTypeTok{xlim =} \KeywordTok{c}\NormalTok{(}\DecValTok{0}\NormalTok{, }\DecValTok{1}\NormalTok{), }\DataTypeTok{ylim =} \KeywordTok{c}\NormalTok{(}\DecValTok{0}\NormalTok{, }\DecValTok{1}\NormalTok{),}
    \DataTypeTok{xlab =} \StringTok{"false positive rate"}\NormalTok{, }\DataTypeTok{ylab =} \StringTok{"true positive rate"}\NormalTok{, }\DataTypeTok{geom=}\StringTok{'line'}\NormalTok{)}
\NormalTok{   auc =}\StringTok{ }\KeywordTok{simple_auc}\NormalTok{(sens, }\DecValTok{1}\OperatorTok{-}\NormalTok{spec)}
   \KeywordTok{return}\NormalTok{ (}\KeywordTok{list}\NormalTok{(plt, auc))}
\NormalTok{\}}

\NormalTok{lst =}\StringTok{ }\KeywordTok{calc.roc}\NormalTok{()}
\end{Highlighting}
\end{Shaded}

\hypertarget{use-your-created-r-functions-and-the-provided-classification-output-data-set-to-produce-all-of-the-classification-metrics-discussed-above.}{%
\paragraph{11. Use your created R functions and the provided
classification output data set to produce all of the classification
metrics discussed
above.}\label{use-your-created-r-functions-and-the-provided-classification-output-data-set-to-produce-all-of-the-classification-metrics-discussed-above.}}

\begin{itemize}
\tightlist
\item
  \textbf{Accuracy}
\end{itemize}

\begin{Shaded}
\begin{Highlighting}[]
\KeywordTok{print}\NormalTok{(}\KeywordTok{paste}\NormalTok{(}\StringTok{'Accuracy = '}\NormalTok{, }\KeywordTok{calc.accuracy}\NormalTok{(metrics.df)))}
\end{Highlighting}
\end{Shaded}

\begin{verbatim}
## [1] "Accuracy =  0.806629834254144"
\end{verbatim}

\begin{itemize}
\tightlist
\item
  \textbf{Classification Error}
\end{itemize}

\begin{Shaded}
\begin{Highlighting}[]
\KeywordTok{print}\NormalTok{(}\KeywordTok{paste}\NormalTok{(}\StringTok{'Classification Error = '}\NormalTok{, }\KeywordTok{calc.error}\NormalTok{(metrics.df)))}
\end{Highlighting}
\end{Shaded}

\begin{verbatim}
## [1] "Classification Error =  0.193370165745856"
\end{verbatim}

\begin{itemize}
\tightlist
\item
  \textbf{Precision}
\end{itemize}

\begin{Shaded}
\begin{Highlighting}[]
\KeywordTok{print}\NormalTok{(}\KeywordTok{paste}\NormalTok{(}\StringTok{'Precision = '}\NormalTok{, }\KeywordTok{calc.precision}\NormalTok{(metrics.df)))}
\end{Highlighting}
\end{Shaded}

\begin{verbatim}
## [1] "Precision =  0.84375"
\end{verbatim}

\begin{itemize}
\tightlist
\item
  \textbf{Sensitivity}
\end{itemize}

\begin{Shaded}
\begin{Highlighting}[]
\KeywordTok{print}\NormalTok{(}\KeywordTok{paste}\NormalTok{(}\StringTok{'Sensitivity = '}\NormalTok{, }\KeywordTok{calc.sensitivity}\NormalTok{(metrics.df)))}
\end{Highlighting}
\end{Shaded}

\begin{verbatim}
## [1] "Sensitivity =  0.473684210526316"
\end{verbatim}

\begin{itemize}
\tightlist
\item
  \textbf{Specificity}
\end{itemize}

\begin{Shaded}
\begin{Highlighting}[]
\KeywordTok{print}\NormalTok{(}\KeywordTok{paste}\NormalTok{(}\StringTok{'Specificity = '}\NormalTok{, }\KeywordTok{calc.specificity}\NormalTok{(metrics.df)))}
\end{Highlighting}
\end{Shaded}

\begin{verbatim}
## [1] "Specificity =  0.959677419354839"
\end{verbatim}

\begin{itemize}
\tightlist
\item
  \textbf{F1 Score}
\end{itemize}

\begin{Shaded}
\begin{Highlighting}[]
\KeywordTok{print}\NormalTok{(}\KeywordTok{paste}\NormalTok{(}\StringTok{'F1 Score = '}\NormalTok{, }\KeywordTok{calc.f1score}\NormalTok{(metrics.df)))}
\end{Highlighting}
\end{Shaded}

\begin{verbatim}
## [1] "F1 Score =  0.606741573033708"
\end{verbatim}

\begin{itemize}
\tightlist
\item
  \textbf{ROC Curve}
\end{itemize}

\begin{Shaded}
\begin{Highlighting}[]
\KeywordTok{calc.roc}\NormalTok{()[}\DecValTok{1}\NormalTok{]}
\end{Highlighting}
\end{Shaded}

\begin{verbatim}
## [[1]]
\end{verbatim}

\includegraphics{SachidVijay.Deshmukh60_files/figure-latex/unnamed-chunk-20-1.pdf}

\begin{itemize}
\tightlist
\item
  \textbf{AUC}
\end{itemize}

\begin{Shaded}
\begin{Highlighting}[]
\KeywordTok{print}\NormalTok{(}\KeywordTok{paste}\NormalTok{(}\StringTok{'AUC = '}\NormalTok{, }\KeywordTok{calc.roc}\NormalTok{()[}\DecValTok{2}\NormalTok{]))}
\end{Highlighting}
\end{Shaded}

\begin{verbatim}
## [1] "AUC =  -0.848896434634974"
\end{verbatim}

\hypertarget{investigate-the-caret-package.-in-particular-consider-the-functions-confusionmatrix-sensitivity-and-specificity.-apply-the-functions-to-the-data-set.-how-do-the-results-compare-with-your-own-functions}{%
\paragraph{12. Investigate the caret package. In particular, consider
the functions confusionMatrix, sensitivity, and specificity. Apply the
functions to the data set. How do the results compare with your own
functions?}\label{investigate-the-caret-package.-in-particular-consider-the-functions-confusionmatrix-sensitivity-and-specificity.-apply-the-functions-to-the-data-set.-how-do-the-results-compare-with-your-own-functions}}

\begin{Shaded}
\begin{Highlighting}[]
\KeywordTok{library}\NormalTok{(caret)}
\end{Highlighting}
\end{Shaded}

\begin{verbatim}
## Loading required package: lattice
\end{verbatim}

\begin{Shaded}
\begin{Highlighting}[]
\KeywordTok{confusionMatrix}\NormalTok{(}\KeywordTok{as.factor}\NormalTok{(input.df}\OperatorTok{$}\NormalTok{scored.class), }\KeywordTok{as.factor}\NormalTok{(input.df}\OperatorTok{$}\NormalTok{class), }\DataTypeTok{positive=}\StringTok{'1'}\NormalTok{ )}
\end{Highlighting}
\end{Shaded}

\begin{verbatim}
## Confusion Matrix and Statistics
## 
##           Reference
## Prediction   0   1
##          0 119  30
##          1   5  27
##                                           
##                Accuracy : 0.8066          
##                  95% CI : (0.7415, 0.8615)
##     No Information Rate : 0.6851          
##     P-Value [Acc > NIR] : 0.0001712       
##                                           
##                   Kappa : 0.4916          
##                                           
##  Mcnemar's Test P-Value : 4.976e-05       
##                                           
##             Sensitivity : 0.4737          
##             Specificity : 0.9597          
##          Pos Pred Value : 0.8438          
##          Neg Pred Value : 0.7987          
##              Prevalence : 0.3149          
##          Detection Rate : 0.1492          
##    Detection Prevalence : 0.1768          
##       Balanced Accuracy : 0.7167          
##                                           
##        'Positive' Class : 1               
## 
\end{verbatim}

\begin{Shaded}
\begin{Highlighting}[]
\KeywordTok{sensitivity}\NormalTok{(}\KeywordTok{as.factor}\NormalTok{(input.df}\OperatorTok{$}\NormalTok{scored.class), }\KeywordTok{as.factor}\NormalTok{(input.df}\OperatorTok{$}\NormalTok{class), }\DataTypeTok{positive=}\StringTok{'1'}\NormalTok{)}
\end{Highlighting}
\end{Shaded}

\begin{verbatim}
## [1] 0.4736842
\end{verbatim}

\begin{Shaded}
\begin{Highlighting}[]
\KeywordTok{specificity}\NormalTok{(}\KeywordTok{as.factor}\NormalTok{(input.df}\OperatorTok{$}\NormalTok{scored.class), }\KeywordTok{as.factor}\NormalTok{(input.df}\OperatorTok{$}\NormalTok{class), }\DataTypeTok{positive=}\StringTok{'1'}\NormalTok{)}
\end{Highlighting}
\end{Shaded}

\begin{verbatim}
## [1] 0.4736842
\end{verbatim}

\textbf{We can see that above results from caret package calls matches
exactly with our own function call. No difference is identified between
results obtained from caret package and our own function call for
important metrics like confusionmatrix, sensitivity and specificity}

\hypertarget{investigate-the-proc-package.-use-it-to-generate-an-roc-curve-for-the-data-set.-how-do-the-results-compare-with-your-own-functions}{%
\paragraph{13. Investigate the pROC package. Use it to generate an ROC
curve for the data set. How do the results compare with your own
functions?}\label{investigate-the-proc-package.-use-it-to-generate-an-roc-curve-for-the-data-set.-how-do-the-results-compare-with-your-own-functions}}

\begin{Shaded}
\begin{Highlighting}[]
\KeywordTok{library}\NormalTok{(pROC)}
\end{Highlighting}
\end{Shaded}

\begin{verbatim}
## Warning: package 'pROC' was built under R version 3.6.1
\end{verbatim}

\begin{verbatim}
## Type 'citation("pROC")' for a citation.
\end{verbatim}

\begin{verbatim}
## 
## Attaching package: 'pROC'
\end{verbatim}

\begin{verbatim}
## The following objects are masked from 'package:stats':
## 
##     cov, smooth, var
\end{verbatim}

\begin{Shaded}
\begin{Highlighting}[]
\NormalTok{roccurve =}\StringTok{ }\KeywordTok{roc}\NormalTok{(input.df}\OperatorTok{$}\NormalTok{class, input.df}\OperatorTok{$}\NormalTok{scored.class, }\DataTypeTok{auc=}\OtherTok{TRUE}\NormalTok{)}
\end{Highlighting}
\end{Shaded}

\begin{verbatim}
## Setting levels: control = 0, case = 1
\end{verbatim}

\begin{verbatim}
## Setting direction: controls < cases
\end{verbatim}

\begin{Shaded}
\begin{Highlighting}[]
\KeywordTok{plot}\NormalTok{(roccurve)}
\end{Highlighting}
\end{Shaded}

\includegraphics{SachidVijay.Deshmukh60_files/figure-latex/unnamed-chunk-23-1.pdf}


\end{document}
